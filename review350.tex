\documentclass[10pt,landscape]{article}
\usepackage{amssymb,amsmath,amsthm,amsfonts}
\usepackage{multicol,multirow}
\usepackage{calc}
\usepackage{ifthen}
\usepackage[landscape]{geometry}
\usepackage[colorlinks=true,citecolor=blue,linkcolor=blue]{hyperref}


\ifthenelse{\lengthtest { \paperwidth = 11in}}
    { \geometry{top=.5in,left=.5in,right=.5in,bottom=.5in} }
	{\ifthenelse{ \lengthtest{ \paperwidth = 297mm}}
		{\geometry{top=1cm,left=1cm,right=1cm,bottom=1cm} }
		{\geometry{top=1cm,left=1cm,right=1cm,bottom=1cm} }
	}
\pagestyle{empty}
\makeatletter
\renewcommand{\section}{\@startsection{section}{1}{0mm}%
                                {-1ex plus -.5ex minus -.2ex}%
                                {0.5ex plus .2ex}%x
                                {\normalfont\large\bfseries}}
\renewcommand{\subsection}{\@startsection{subsection}{2}{0mm}%
                                {-1explus -.5ex minus -.2ex}%
                                {0.5ex plus .2ex}%
                                {\normalfont\normalsize\bfseries}}
\renewcommand{\subsubsection}{\@startsection{subsubsection}{3}{0mm}%
                                {-1ex plus -.5ex minus -.2ex}%
                                {1ex plus .2ex}%
                                {\normalfont\small\bfseries}}
\makeatother
\setcounter{secnumdepth}{0}
\setlength{\parindent}{0pt}
\setlength{\parskip}{0pt plus 0.5ex} %
% -----------------------------------------------------------------------

\title{MATH 350: Review}

\begin{document}

\raggedright
\footnotesize

\begin{center}
     \Large{\textbf{MATH 350; Midterm Review.}} \\
\end{center}
\begin{multicols}{3}
\setlength{\premulticols}{1pt}
\setlength{\postmulticols}{1pt}
\setlength{\multicolsep}{1pt}
\setlength{\columnsep}{2pt}

\section{Connectivity}
\textbf{Handshaking lemma.} For any graph \( G \), \( \sum_{v \in V(G)}^{}\deg v = 2|E(G)|. \) \\
\textbf{Lemma 2.1.} If there exists a walk in \( G \) with ends \( u,v \) then there exists a path in \( G \) with these ends. \\
\textbf{Lemma 2.2.} A graph \( G \) is not connected iff there exists a partition \( (X,Y) \) of \( V(G) \) such that no edge of \( G \) has one end in \( X \) and the other in \( Y \). \\
\textbf{Lemma 2.3.} If \( H_1, H_2 \) are connected subgraphs of \( G \) and \( V(H_1) \cap V(H_2) \neq \emptyset  \), then \( H_1 \cup H_2 \) is connected. \\
\textbf{Lemma 2.4.} Every vertex of \( G \) belongs to a unique connected component. \\
\textbf{Lemma 2.5.} A subgraph \( H \subseteq G \) is a connected component of \( G \) iff \( H \) is connected, and if \( e \in E(G) \) has an end in \( V(H) \) then \( e \in E(H). \) \\
\textbf{Lemma 2.6.} Let \( e \in E(G) \) with ends \( u,v \). Then exactly one of the following holds.
\begin{enumerate}
	\item \( e \) is a cut edge, \( u \) and \( v \) belong to different conneected components of \( G \setminus e \), and \( \mbox{comp}(G \setminus e) = \mbox{comp}(G) + 1.\)
	\item \( e \) is not a cut edge, \( u \) and \( v \) belong to the same connected component of \( G \setminus e \), and \( \mbox{comp}(G\setminus e ) = \mbox{comp}(G). \) 
\end{enumerate}
\section{Trees \& Forests}
\textbf{Lemma 3.1.} Let \( F \) be a non-null forest. Then \( \mbox{comp}(F) = |V(F)| - |E(F)| \). If \( F \) is a tree, then \( |V(F)| = |E(F)| + 1 \). \\
\textbf{Lemma 3.2.} Let \( T \) tree with \( |V(T)| \geq 2 \). Let \( X \) be the set of leaves of \( T \) and \( Y \) be the set of vertices of degree \( \geq 3 \). Then \( |X| \geq |Y| + 2 \). In particular, \( T \) has at least 2 leaves. \\
\textbf{Lemma 3.3.} If a tree \( T \) has exactly two leaves \( u \) and \( v \), then \( T \) is a path with ends \( u \) and \( v \). \\
\textbf{Lemma 3.4.} Let \( v \) be a leaf in a tree \( T \). Then \( T \setminus v \) is a tree. \\
\textbf{Lemma 3.5.} Let \( v \) be a leaf in a graph \( G \). If \( G \setminus v \) is a tree then \( G \) is a tree. \\
\textbf{Lemma 3.6.} Let \( T \) be a tree, \( u,v \in V(T) \). Then there exists a unique path in \( T \) with ends \( u,v\).
\section{Spanning Trees}
\textbf{Lemma 4.1.} Let \( G \) be a connected, non-null graph. Let \( H \subseteq G \) be chosen minimal such that \( V(H) = V(G) \) and \( H \) is connected. Then \( H \) is a spanning tree of \( G \). \\
\textbf{Lemma 4.2.} Let \( G \) be a connected, non-null graph. Let \( H \subseteq G	\) be chosen maximal such that \( H \) has no cycles. Then \( H \) is a spanning tree of \( G \). \\
\textbf{Lemma 4.3.} Let \( T \) be a spanning tree of \( G \). Let \( f \in E(G) \setminus E(T) \). Then there exists a unique fundamental cycle of \( f \) with respect to \( T \). \\
\textbf{Lemma 4.4.} Let \( T \) be a spanning tree of \( G \), \( f \in E(G) \setminus E(T) \), and \( C \) be the fundamental cycle of \( f \) with respect to \( T \). Let \( T' = (T+f) \setminus e \) be the graph obtained from \( T \) by adding \( f \) and deleting some \( e \in E(C). \) Then \( T' \) is a spanning tree of \( G \). \\
\textbf{Corollary 4.5.} Let \( G,T,f,C,e \) be as in Lemma 4.4. Let \( w : E(G) \to \mathbb{R}_+ \). If \( T \) is \( \mbox{mst}(G,w) \), then \( w(f) \geq w(e). \) \\
\textbf{Theorem 4.6.} Let \( G \) be a graph, \( w :E(G) \to \mathbb{R}_+ \) be such that \( w(e) \neq w(f) \) for any \( e,f \in E(G) \) with \( e \neq f \). Let \( T \) be \( \mbox{mst}(G,w) \) and \( E(T) = \{ e_1,e_2,\hdots ,e_k \}  \) be such that \( w(e_1) < \cdots < w(e_k) \). Then for every \( i \) with \( 1 \leq i \leq k \), \( e_i \) is the edge of minimum weight subject to \( e_i \notin \{ e_1, \hdots , e_{i-1}  \}  \) and \( \{ e_1, \hdots , e_i \} \) does not contain the edge set of a cycle. \\
\textbf{Theorem 4.7.} Kruskal's algorithm outputs \( \mbox{mst}(G,w) \). \\
\textbf{Cayley's formula.} The complete graph on \( n \) vertices has \( n^{n-2}  \) spanning trees. 
\section{Euler Tours \& Hamiltonian Cycles}
\textbf{Facts.} A walk uses two edges incident to a vertex each time this vertex occurs in the walk (except for ends); thus if an Eulerian trail exists, then at most two vertices odd degree; if an Euler tour exists, then all vertices must have even degree. \\ 
\textbf{Lemma 5.1.} Let \( E(G) \neq \emptyset  \) and suppose \( G \) has no leaves. Then \( G \) contains a cycle. \\
\textbf{Lemma 5.2.} Let \( G \) be a graph such that every vertex of \( G \) has even degree. Then there exist cycles \( C_1, \hdots , C_k \) in \( G \) such that \( (E(C_1), \hdots , E(C_k)) \) is a partition of \( E(G). \) \\
\textbf{Euler's theorem.} If \( G \) is a connected graph such that the degree of every vertex of \( G \) is even, then \( G \) has an Euler tour. \\
\textbf{Corollary 5.4.} If \( G \) is a connected graph such that \( G \) contains at most two vertices of odd degree, then \( G \) has an Eulerian trail. \\
\textbf{Lemma 5.5.} Let \( G \) be a graph. If there exists \( X \subseteq V(G) \) with \( X \neq \emptyset  \) such that \( G \setminus X \) has more than \( |X| \) components, then \( G \) has no Hamiltonian cycle. \\
\textbf{Dirac.} Let \( G \) be a simple graph on \( n \geq 3\) vertices. Suppose that for every pair of non-adjacent vertices, \( u ,v \in V(G) \), \( \deg u + \deg v \geq n \). Then \( G \) has a Hamiltonian cycle. \\
\textbf{Corollary 5.7.} Let \( G \) be a simple graph with \( n \geq 3 \) vertices. Suppose that either
\begin{itemize}
	\item \( \deg v \geq n / 2 \) for all \( v \in V(G) \); or
	\item \(|E(G)| \geq \binom{n}{2} -n +3.\)
	
	
\end{itemize}
Then \( G \) has a Hamiltonian cycle.
\section{Bipartite Graphs}
\textbf{Lemma 6.1.} Trees are bipartite. \\
\textbf{Theorem 6.2.} Let \( G \) be a graph. TFAE:
\begin{enumerate}
	\item \( G \) is bipartite;
	\item \( G \) contains no closed walk of odd length;
	\item \( G \) contains no odd cycle.
\end{enumerate}

\section{Matchings}
\textbf{Fact.} \( \nu (G) \leq \frac{1}{2} |V(G)|. \) \\
\textbf{Lemma 7.1.} Let \( G \) be a loopless graph. Then \( \nu (G) \leq \tau (G) \leq 2 \nu(G) \). \\
\textbf{Lemma 7.2.} A matching \( M \) in \( G \) has maximum size iff there does not exist an \( M \)-augmenting path in \( G \). \\
\textbf{Konig.} If \( G \) is bipartite, then \( \nu (G) = \tau (G) \). \\
\textbf{Theorem 7.4.} Let \( d \geq 1 \) be an integer, let \( G \) bipartite such that \( \deg v = d \) for every \( v \in  V(G) \). Then \( G \) has a perfect matching. \\
\textbf{Hall.} Let \( G \) bipartite with bipartition \( (A,B) \). Then \( G \) has a matching \( M \) covering \( A \) iff \( |N(S)| \geq |S| \) for every \( S \subseteq A \). \\
\textbf{Menger.} Let \( s,t \in V(G) \) be a pair of distinct non-adjacent vertices of \( G \), and let \( k \geq 1 \) be an integer. Then exactly one of the following holds.
\begin{enumerate}
	\item There exist pairwise internally disjoint paths \( P_1, \hdots , P_k \) in \( G \) with ends \( s,t \).
	\item There exists a separation \( (A,B) \) of \( G \) such that \( s \in A \setminus B \), \( t \in B \setminus A \) of order \( < k \).
\end{enumerate}
\textbf{Theorem 8.2.} Let \( Q, R \subseteq V(G) \), \( k \geq 1 \) an integer. Then exactly one of the following holds. (1) There exist pairwise disjoint paths \( P_1, \hdots , P_k \) in \( G \) each with one end in \( Q \) and another in \( R \). (2) There exists a separation \( (A,B) \) of \( G \) of order \( < k \) such that \( Q \subseteq A \) and \( R \subseteq B \). \\
\textbf{Corollary 8.3.} Let \( G \) be a \( k \)-connected graph, \( s,t \in V(G) \) be distinct. Then there exist paths \( P_1, \hdots , P_k \) in \( G \) from \( s \) to \( t \) that are internally disjoint. \\
\textbf{Menger 2.} Let \( s, t \in V(G) \) be distinct and \( k \geq 1 \). Then exactly one of the following holds. (1) There exist pairwise internally disjoint paths \( P_1, \hdots , P_k \) in \( G \) with ends \( s,t \). (2) There exists \(  X \subseteq V(G) \) such that \( s \in X \), \( t \in V(G) \setminus X \), and \( |\delta (X)| < k \).
\section{Directed Graphs \& Network Flows}
\textbf{Lemma 9.1.} Let \( G \) be a digraph. Let \( s, t \in V(G) \). Then there does not exist a directed path in \( G \) from \( s \) to \( t \) iff there exists \( X \subseteq V(G) \) such that \( s \in X \), \( t \in V(G) \setminus X \), and \( \delta ^{+}(X) = \emptyset   \). \\
\textbf{Lemma 9.2.} Let \( \varphi  \) be an \( (s,t) \)-flow on a digraph \( G \) with value \( k \). Then for any \( X \subseteq V(G) \) such that \( s \in X \), \( t \in V(G) \setminus X \), we have \[\sum_{e \in \delta ^{+}(X) }^{}\varphi (e) - \sum_{e \in \delta ^{-}(X)}^{}\varphi (e) = k.\]
\section{Definitions}
\begin{enumerate}
	\item \textbf{Matching.} A matching \( M \subseteq E(G) \) in a graph \( G \) is a collection of non-loop edges so that every vertex of \( G \) is incident to at most one edge of \( M \).
	\item \textbf{M-alternating path.} An \( M \)-alternating path \( P \) is a path whose edges alternate between belonging to \( M \) and belonging to \( E(G) \setminus M \). Equivalently, every internal vertex of the path is incident to one edge in \( M \cap E(P) \) and one edge of \( (E(G) \setminus M) \cap E(P). \)
	\item \textbf{M-augmenting path.} A path \( P \) is \( M \)-augmenting if \( |P| \geq 1 \), it is \( M \)-alternating, and the ends of \( P \) belong to no edge of \( M \) (whether in \( P \) or not).
	\item \textbf{Vertex cover.} A vertex cover is a subset \( X \subseteq V(G) \) if every edge of \( G \) has an end in \( X \).
	\item \( \nu (G) \) is the \textbf{matching number} of \( G \), i.e. the maximum size of a matching in \( G \). \( \tau (G) \) is the minimum size of a vertex cover in \( G \).
	\item \textbf{Cover.} A matching \( M \) covers \( Y \subseteq V(G) \) is every vertex in \( Y \) is incident to an edge of \( M \).
	\item \textbf{Perfect matching.} \( M \) is a perfect matching if it covers \( V(G) \).
	\item \textbf{Separation.} A pair \( (A,B) \), \( A,B \subseteq V(G) \) is called a separation if \( A \cup B = V(G) \) and no edge of \( G \) has one end in \( A \setminus B \) and another in \( B \setminus A \), i.e. any edge from \( A \) to \( B \) has an end in \( A \cap B \).
	\item \textbf{k-Connected.} A graph if \( k \)-connected if \( |V(G)| \geq k +1  \) and \( G \setminus X \) is connected for every \( X \subseteq V(G) \) with \( |X| \leq k+1. \)
	\item For \( X \subseteq V(G) \), let \( \delta (X) \) denote the set of all edges of \( G \) with one end in \( X \) and another in \( V(G) \setminus X \). Let \( N(X) \) the set of all neighbours of vertices in \( X \).
	\item \textbf{Line graph.} The line graph \( L(G) \) of \( G \) has \( V(L(G)) = E(G) \), and two vertices of \( L(G) \) are adjacent iff the edges they represent share an end in \( G \).
	\item \textbf{Directed graph.} A directed graph (digraph) is a graph where for every edge, one of its ends is chosen as a head and the other as a tail. An edge is said to be directed from its tail to its head.
	\item \textbf{Directed path.} A directed path from \( s \) to \( t \) in a digraph is a path in which every edge is traversed from its tail to its head as we follow the path from \( s \) to \( t \).
	\item For a set \( X \subseteq V(G) \), we let \( \delta ^{+} (X) \) denote the set of all edges with tail in \( X \) and head in \( V(G) \setminus X \); we let \( \delta ^{-} (X) = \delta ^{+} (V(G) \setminus X)  \).
	\item \textbf{(s,t)-Flow.} Let \( G \) be a digraph, \( s , t \in V(G) \). A function \( \varphi : E(G) \to \mathbb{R}^{+}  \) is an \( (s,t) \)-flow if for every \( v \in V(G) \setminus \{ s,t\}  \), \( \sum_{e \in \delta ^{-1}(v) }^{}\varphi (e)= \sum_{e \in \delta ^{-} (v)}^{} \varphi (e) \). The \textbf{value} of \( \varphi  \) is \( \sum_{e \in \delta ^{+} (s)}^{}\varphi (e) - \sum_{e \in \delta ^{-} (s)}^{}\varphi (e). \) 
	
	
	
	
	
	
	
	
	
	
	
	
	
	
\end{enumerate}
\end{document}


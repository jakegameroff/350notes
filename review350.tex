\documentclass[10pt,landscape]{article}
\usepackage{amssymb,amsmath,amsthm,amsfonts}
\usepackage{multicol,multirow}
\usepackage{calc}
\usepackage{ifthen}
\usepackage[landscape]{geometry}
\usepackage[colorlinks=true,citecolor=blue,linkcolor=blue]{hyperref}


\ifthenelse{\lengthtest { \paperwidth = 11in}}
    { \geometry{top=.5in,left=.5in,right=.5in,bottom=.5in} }
	{\ifthenelse{ \lengthtest{ \paperwidth = 297mm}}
		{\geometry{top=1cm,left=1cm,right=1cm,bottom=1cm} }
		{\geometry{top=1cm,left=1cm,right=1cm,bottom=1cm} }
	}
\pagestyle{empty}
\makeatletter
\renewcommand{\section}{\@startsection{section}{1}{0mm}%
                                {-1ex plus -.5ex minus -.2ex}%
                                {0.5ex plus .2ex}%x
                                {\normalfont\large\bfseries}}
\renewcommand{\subsection}{\@startsection{subsection}{2}{0mm}%
                                {-1explus -.5ex minus -.2ex}%
                                {0.5ex plus .2ex}%
                                {\normalfont\normalsize\bfseries}}
\renewcommand{\subsubsection}{\@startsection{subsubsection}{3}{0mm}%
                                {-1ex plus -.5ex minus -.2ex}%
                                {1ex plus .2ex}%
                                {\normalfont\small\bfseries}}
\makeatother
\setcounter{secnumdepth}{0}
\setlength{\parindent}{0pt}
\setlength{\parskip}{0pt plus 0.5ex} %
% -----------------------------------------------------------------------

\title{MATH 350: Review}

\begin{document}

\raggedright
\footnotesize

\begin{center}
	\Large{\textbf{MATH 350 Results.}} \\
\end{center}
\begin{multicols}{3}
\setlength{\premulticols}{1pt}
\setlength{\postmulticols}{1pt}
\setlength{\multicolsep}{1pt}
\setlength{\columnsep}{2pt}

\section{Connectivity}
\textbf{Handshaking lemma.} For any graph \( G \), \( \sum_{v \in V(G)}^{}\deg v = 2|E(G)|. \) \\
\textbf{Lemma 2.1.} If there exists a walk in \( G \) with ends \( u,v \) then there exists a path in \( G \) with these ends. \\
\textbf{Lemma 2.2.} A graph \( G \) is not connected iff there exists a partition \( (X,Y) \) of \( V(G) \) such that no edge of \( G \) has one end in \( X \) and the other in \( Y \). \\
\textbf{Lemma 2.3.} If \( H_1, H_2 \) are connected subgraphs of \( G \) and \( V(H_1) \cap V(H_2) \neq \emptyset  \), then \( H_1 \cup H_2 \) is connected. \\
\textbf{Lemma 2.4.} Every vertex of \( G \) belongs to a unique connected component. \\
\textbf{Lemma 2.5.} A subgraph \( H \subseteq G \) is a connected component of \( G \) iff \( H \) is connected, and if \( e \in E(G) \) has an end in \( V(H) \) then \( e \in E(H). \) \\
\textbf{Lemma 2.6.} Let \( e \in E(G) \) with ends \( u,v \). Then exactly one of the following holds.
\begin{enumerate}
	\item \( e \) is a cut edge, \( u \) and \( v \) belong to different conneected components of \( G \setminus e \), and \( \mbox{comp}(G \setminus e) = \mbox{comp}(G) + 1.\)
	\item \( e \) is not a cut edge, \( u \) and \( v \) belong to the same connected component of \( G \setminus e \), and \( \mbox{comp}(G\setminus e ) = \mbox{comp}(G). \) 
\end{enumerate}
\section{Trees \& Forests}
\textbf{Lemma 3.1.} Let \( F \) be a non-null forest. Then \( \mbox{comp}(F) = |V(F)| - |E(F)| \). If \( F \) is a tree, then \( |V(F)| = |E(F)| + 1 \). \\
\textbf{Lemma 3.2.} Let \( T \) tree with \( |V(T)| \geq 2 \). Let \( X \) be the set of leaves of \( T \) and \( Y \) be the set of vertices of degree \( \geq 3 \). Then \( |X| \geq |Y| + 2 \). In particular, \( T \) has at least 2 leaves. \\
\textbf{Lemma 3.3.} If a tree \( T \) has exactly two leaves \( u \) and \( v \), then \( T \) is a path with ends \( u \) and \( v \). \\
\textbf{Lemma 3.4.} Let \( v \) be a leaf in a tree \( T \). Then \( T \setminus v \) is a tree. \\
\textbf{Lemma 3.5.} Let \( v \) be a leaf in a graph \( G \). If \( G \setminus v \) is a tree then \( G \) is a tree. \\
\textbf{Lemma 3.6.} Let \( T \) be a tree, \( u,v \in V(T) \). Then there exists a unique path in \( T \) with ends \( u,v\).
\section{Spanning Trees}
\textbf{Lemma 4.1.} Let \( G \) be a connected, non-null graph. Let \( H \subseteq G \) be chosen minimal such that \( V(H) = V(G) \) and \( H \) is connected. Then \( H \) is a spanning tree of \( G \). \\
\textbf{Lemma 4.2.} Let \( G \) be a connected, non-null graph. Let \( H \subseteq G	\) be chosen maximal such that \( H \) has no cycles. Then \( H \) is a spanning tree of \( G \). \\
\textbf{Lemma 4.3.} Let \( T \) be a spanning tree of \( G \). Let \( f \in E(G) \setminus E(T) \). Then there exists a unique fundamental cycle of \( f \) with respect to \( T \). \\
\textbf{Lemma 4.4.} Let \( T \) be a spanning tree of \( G \), \( f \in E(G) \setminus E(T) \), and \( C \) be the fundamental cycle of \( f \) with respect to \( T \). Let \( T' = (T+f) \setminus e \) be the graph obtained from \( T \) by adding \( f \) and deleting some \( e \in E(C). \) Then \( T' \) is a spanning tree of \( G \). \\
\textbf{Corollary 4.5.} Let \( G,T,f,C,e \) be as in Lemma 4.4. Let \( w : E(G) \to \mathbb{R}_+ \). If \( T \) is \( \mbox{mst}(G,w) \), then \( w(f) \geq w(e). \) \\
\textbf{Kruskal's Algorithm:} Input a connected non-null graph \( G \) and \( w : E(G) \to \mathbb{R}_{+}  \). For \( i = 1, 2, \hdots , |V(G)| - 1 \), let \( e_{i} \in E(G) \) be chosen with \( w(e_{i} ) \) minimum such that \( e_{i} \notin \{ e_1, e_2, \hdots , e_{i  -   1}  \}  \) and \( \{ e_1, e_2, \hdots , e_{i}  \}  \) does not contain the edge set of a cycle. Output is a tree \( T \) with \( V(T) = V(G) \) and \( E(T) = \{ e_1, \hdots , e_{|V(G)| - 1}  \}  \). \\
\textbf{Theorem 4.6.} Let \( G \) be a graph, \( w :E(G) \to \mathbb{R}_+ \) be such that \( w(e) \neq w(f) \) for any \( e,f \in E(G) \) with \( e \neq f \). Let \( T \) be \( \mbox{mst}(G,w) \) and \( E(T) = \{ e_1,e_2,\hdots ,e_k \}  \) be such that \( w(e_1) < \cdots < w(e_k) \). Then for every \( i \) with \( 1 \leq i \leq k \), \( e_i \) is the edge of minimum weight subject to \( e_i \notin \{ e_1, \hdots , e_{i-1}  \}  \) and \( \{ e_1, \hdots , e_i \} \) does not contain the edge set of a cycle. \\
\textbf{Theorem 4.7.} Kruskal's algorithm outputs \( \mbox{mst}(G,w) \). \\
\textbf{Cayley's formula.} The complete graph on \( n \) vertices has \( n^{n-2}  \) spanning trees. 
\section{Euler Tours \& Hamiltonian Cycles}
\textbf{Facts.} A walk uses two edges incident to a vertex each time this vertex occurs in the walk (except for ends); thus if an Eulerian trail exists, then at most two vertices odd degree; if an Euler tour exists, then all vertices must have even degree. \\ 
\textbf{Lemma 5.1.} Let \( E(G) \neq \emptyset  \) and suppose \( G \) has no leaves. Then \( G \) contains a cycle. \\
\textbf{Lemma 5.2.} Let \( G \) be a graph such that every vertex of \( G \) has even degree. Then there exist cycles \( C_1, \hdots , C_k \) in \( G \) such that \( (E(C_1), \hdots , E(C_k)) \) is a partition of \( E(G). \) \\
\textbf{Euler's theorem.} If \( G \) is a connected graph such that the degree of every vertex of \( G \) is even, then \( G \) has an Euler tour. \\
\textbf{Corollary 5.4.} If \( G \) is a connected graph such that \( G \) contains at most two vertices of odd degree, then \( G \) has an Eulerian trail. \\
\textbf{Lemma 5.5.} Let \( G \) be a graph. If there exists \( X \subseteq V(G) \) with \( X \neq \emptyset  \) such that \( G \setminus X \) has more than \( |X| \) components, then \( G \) has no Hamiltonian cycle. \\
\textbf{Dirac.} Let \( G \) be a simple graph on \( n \geq 3\) vertices. Suppose that for every pair of non-adjacent vertices, \( u ,v \in V(G) \), \( \deg u + \deg v \geq n \). Then \( G \) has a Hamiltonian cycle. \\
\textbf{Corollary 5.7.} Let \( G \) be a simple graph with \( n \geq 3 \) vertices. Suppose that either
\begin{itemize}
	\item \( \deg v \geq n / 2 \) for all \( v \in V(G) \); or
	\item \(|E(G)| \geq \binom{n}{2} -n +3.\)
	
	
\end{itemize}
Then \( G \) has a Hamiltonian cycle.
\section{Bipartite Graphs}
\textbf{Lemma 6.1.} Trees are bipartite. \\
\textbf{Theorem 6.2.} Let \( G \) be a graph. TFAE:
\begin{enumerate}
	\item \( G \) is bipartite;
	\item \( G \) contains no closed walk of odd length;
	\item \( G \) contains no odd cycle.
\end{enumerate}

\section{Matchings}
\textbf{Fact.} \( \nu (G) \leq \frac{1}{2} |V(G)|. \) \\
\textbf{Lemma 7.1.} Let \( G \) be a loopless graph. Then \( \nu (G) \leq \tau (G) \leq 2 \nu(G) \). \\
\textbf{Lemma 7.2.} A matching \( M \) in \( G \) has maximum size iff there does not exist an \( M \)-augmenting path in \( G \). \\
\textbf{Konig.} If \( G \) is bipartite, then \( \nu (G) = \tau (G) \). \\
\textbf{Konig rephrased.} Let \( G \) bipartite. \( G \) has a matching \( M \) with \( |M| \geq k \iff \) There does not exist \( X \subseteq V(G) \) with \( |X| < k \) such that every edge of \( G \) has an end in \( X \), i.e. there is no vertex cover with \( < k \) vertices. \\
\textbf{Theorem 7.4.} Let \( d \geq 1 \) be an integer, let \( G \) bipartite such that \( \deg v = d \) for every \( v \in  V(G) \). Then \( G \) has a perfect matching. \\
\textbf{Hall.} Let \( G \) bipartite with bipartition \( (A,B) \). Then \( G \) has a matching \( M \) covering \( A \) iff \( |N(S)| \geq |S| \) for every \( S \subseteq A \). \\
\textbf{Menger.} Let \( s,t \in V(G) \) be a pair of distinct non-adjacent vertices of \( G \), and let \( k \geq 1 \) be an integer. Then exactly one of the following holds.
\begin{enumerate}
	\item There exist pairwise internally disjoint paths \( P_1, \hdots , P_k \) in \( G \) with ends \( s,t \).
	\item There exists a separation \( (A,B) \) of \( G \) such that \( s \in A \setminus B \), \( t \in B \setminus A \) of order \( < k \).
\end{enumerate}
\textbf{Theorem 8.2.} Let \( Q, R \subseteq V(G) \), \( k \geq 1 \) an integer. Then exactly one of the following holds. (1) There exist pairwise disjoint paths \( P_1, \hdots , P_k \) in \( G \) each with one end in \( Q \) and another in \( R \). (2) There exists a separation \( (A,B) \) of \( G \) of order \( < k \) such that \( Q \subseteq A \) and \( R \subseteq B \). \\
\textbf{Corollary 8.3.} Let \( G \) be a \( k \)-connected graph, \( s,t \in V(G) \) be distinct. Then there exist paths \( P_1, \hdots , P_k \) in \( G \) from \( s \) to \( t \) that are internally disjoint. \\
\textbf{Menger 2.} Let \( s, t \in V(G) \) be distinct and \( k \geq 1 \). Then exactly one of the following holds. (1) There exist paths \( P_1, \hdots , P_k \) in \( G \) with ends \( s,t \) and such that for \( 1 \leq i < j \leq n \), \( E(P_i) \cap E(P_j) = \emptyset  \) (pairwise edge disjoint). (2) There exists \(  X \subseteq V(G) \) such that \( s \in X \), \( t \in V(G) \setminus X \), and \( |\delta (X)| < k \).
\section{Directed Graphs \& Network Flows}
\textbf{Lemma 9.1.} Let \( G \) be a digraph. Let \( s, t \in V(G) \). Then there does not exist a directed path in \( G \) from \( s \) to \( t \) iff there exists \( X \subseteq V(G) \) such that \( s \in X \), \( t \in V(G) \setminus X \), and \( \delta ^{+}(X) = \emptyset   \). \\
\textbf{Lemma 9.2.} Let \( \varphi  \) be an \( (s,t) \)-flow on a digraph \( G \) with value \( k \). Then for any \( X \subseteq V(G) \) such that \( s \in X \), \( t \in V(G) \setminus X \), we have \[\sum_{e \in \delta ^{+}(X) }^{}\varphi (e) - \sum_{e \in \delta ^{-}(X)}^{}\varphi (e) = k.\]
\textbf{Lemma 9.3.} Let \( \varphi  \) be an integral \( (s,t) \)-flow on a digraph \( G \) withvalue \( k \geq 0 \). Then there exist directed paths \( P_1, \hdots , P_{k}  \) from \( s \) to \( t \) such that every edge of \( G \) belongs to at most \( \varphi (e) \) of these paths (paths may not be unique). \\
\textbf{Lemma 9.4.} Let \( \varphi \) be an integral \( c \)-admissible \( (s,t) \)-flow on \( G \) with value \( k \). If there exists a \( \varphi \)-augmenting path \( P \) from \( s \) to \( t \) then there exists an integral \( c \)-admissible \( (s,t) \)-flow on \( G \) of value \( k + 1 \). \\
\textbf{Theorem 9.5 (Max flow min cut).} Let \( k \in \mathbb{N}  \). Then exactly one of the following holds:
\begin{enumerate}
	\item There exists an integral \( c \)-admissible \( (s,t) \)-flow of value \( \geq k \).
	\item There exists \( X \subseteq V(G) \), \( s \in X \), \( t \in V(G) \setminus X \) such that \( \sum_{e \in \delta ^{+} (X)}^{} c(e) < k \).
\end{enumerate}
\section{Independence \& Cliques}
\textbf{Properties:} For \( P_{n}  \): \( \nu = \lfloor n/2 \rfloor  \), \( \tau = \lfloor n / 2 \rfloor \), \( \alpha =\lceil n/2 \rceil \), \( \rho = \lceil n / 2 \rceil  \); for \( C_{n}  \): \( \nu = \lfloor n / 2 \rfloor  \), \( \tau = \lceil n / 2 \rceil  \), \( \alpha = \lfloor n / 2 \rfloor  \), \( \rho = \lceil n / 2 \rceil  \); for \( K_{n}  \): \( \nu = \lfloor n / 2 \rfloor  \), \( \tau = n - 1 \), \( \alpha = 1 \), \( \rho = \lceil n / 2 \rceil  \). \\
\textbf{ALSO:} \( \rho (G) \geq \alpha(G) \) and \( \rho (G) \geq |V(G)| / 2 \). \\
\textbf{Lemma 10.1.} For any graph \( G \), \( \alpha(G) + \tau(G) = |V(G)| \). \\
\textbf{Lemma 10.2.} Let \( G \) be simple and such that every vertex of \( G \) is incident to an edge, then \( \nu (G) + \rho (G) = |V(G)| \). \\
\textbf{Corollary 10.3.} Let \( G \) be a simple bipartite graph such that every vertex is incident to an edge, then \( \alpha(G) = \rho (G) \). \\
\section{Ramsey Numbers}
\textbf{Properties:} \( R(s,t) = R(t,s) \), \( R(1,t) = 1 \), \( R(2, t) = t \), \( R(3,3) = 6 \), \( R(3,4) = 9 \). \\
\textbf{Ramsey Theorem.} \( R(s,t) \) exists for every \( s,t \geq 1 \) and \( R(s,t) \leq R(s- 1, t) + R(s, t-1) \). \\
\textbf{Corollary 10.5.} For \( s, t \geq 1 \), \( R(s,t) \leq \binom{s + t - 2}{s - 1}  \).
\textbf{Lemma 10.6.} If \( \binom{N}{s} 2^{1 - \binom{s}{2} } < 1 \) then there exists a simple graph \( G \) with \( |V(G)| = N \) and no clique or independent set of size \( s \) (so \( R(s,s) > N \) ). \\
\textbf{Theorem 10.7.} For \( s \geq 2 \), \( R(s,s) \geq 2^{s / 2} = (\sqrt{2} )^{s}  \). \\
\textbf{Theoreom 10.8.} \( R_{k} (s_1, \hdots , s_{k} ) \) exists for all \( k, s_1, \hdots , s_{k} \geq 1  \). Furthermore, \( R_{k} (s_1, \hdots , s_{k} ) \leq R_{k - 1} (R_{2} (s_1, s_2), s_3, \hdots , s_{k} )  \). \\
\textbf{Schur.} For every \( k \geq 1 \) there exists \( N \geq 1 \) such that in every coloring of \( \{ 1, 2, \hdots , N \}  \) in \( k \) colors there exist \( x,y,z \) of the same color and not necessarily distinct such that \( x + y = z \). \\
\section{Vertex Coloring}
\textbf{Properties.} \( G \) is 1-colorable iff \( G \) is edgeless; \( G \) is 2-colorable iff \( G \) is bipartite. \( G \) is 1-degenerate iff forest. \( \chi_{} (K_{n} ) = n \) \\
\textbf{Lemma 11.1.} Let \( G \) be loopless, then
\begin{enumerate}
	\item \( \chi(G) \geq \omega (G) \) 
	\item \( \chi_{} (G)  \geq \lceil \frac{|V(G)|}{\alpha(G)}  \rceil \).
\end{enumerate}
\textbf{Lemma 11.2.} If \( G \) is loopless and \( k \)-degenerate then \( \chi_{} (G) \leq k + 1 \). In particular, \( \chi_{} (G) \leq \Delta (G) + 1 \). \\
\textbf{Greedy coloring algorithm.} Input a loopless graph \( G \) and an ordering \( (v_1, v_2, \hdots , v_{n} ) \) of \( V(G) \). Outputs a \( k \)-coloring of \( G \) for some \(  k\). 
\begin{enumerate}
	\item Colour \( v_1 \) using color 1
	\item once \( \{ v_1, \hdots , v_{i - 1}  \}  \) receive colors, color \( v_{i}  \) with the smallest integer \( l \in \mathbb{N}  \) that is not a color of any of the neighbours of \( v_{i}  \) so far.
\end{enumerate}
\textbf{Brooks.} Let \( G \) be connected loopless s.t. \( G \) is not complete and not an odd cycle. Then \( \chi (G) \leq \Delta (G) \). \\
\section{Edge Coloring}
\textbf{Properties.} \( \chi_{} '(K_{4} ) = 3 \), \( \chi \ ' ( C_{2k+1})  = 3\). 1-factor iff perfect matching. \\
\textbf{Lemma 12.1.} \( \Delta (G) \leq \chi_{} ' (G) \leq 2 \Delta (G) - 1 \), for any loopless graph \( G \) with \( \Delta (G) \geq 1 \). \\
\textbf{Lemma 12.2.} Let \( G \) be a graph with \( \Delta (G) \leq k \). Then there exists a \( k \)-regular graph \( H \) such that \( G \) is a subgraph of \( H \). Moreover, if \( G \) is loopless (resp. bipartite) then \( H \) can be chosen to be loopless (resp. bipartite). (If \( G \) is simple then \( H \) can be chosen to be simple). \\
\textbf{Theorem 12.3.} If \( G \) is bipartite then \( \chi_{} ' (G) = \Delta (G) \). \\
\textbf{Lemma 12.4.} Let \( G \) be a loopless \( 2k \)-regular graph. Then \( E(G) \) can be partitioned into \( k \) 2-factors. \\
\textbf{Shannon.} Let \( G \) be a loopless graph then \( \chi_{} ' (G) \leq 3 \lceil \frac{\Delta (G)}{2}  \rceil  \). \\
\textbf{Vizing.} If \( G \) simple then \( \chi_{} ' (G) \leq \Delta (G) + 1 \).
\section{Minors}
\textbf{Facts.} No loop minor iff forest; no \( K_2 \) minor iff no \( K_2 \) subgraph; no \( K_3 \) minor iff no cycles of length \( \geq 3 \) iff forest with added loops and parallel edges. \\
\textbf{Hadwiger:} For \( 0 \leq t \geq 5 \) if \( G \) is a loopless graph with no \( K_{t+1}  \) minor then \( \chi_{} (G) \leq t \). Also, \( K_{t+1}  \) subgraph implies \( \chi_{} (G) \geq t + 1 \). \\
\textbf{Subdivision.} If \( G \) is a subdivision of \( H \) then \( H \) is a minor of \( G \) (converse does not hold). \\
\textbf{Lemma 13.1.} If \( G \) is 3-connected then \( G \) has a \( K_4 \) minor. \\
\textbf{Lemma 13.2.} Let \( G \) be a simple graph with no \( K_4 \) minoir. Let \( X \) be a clique in \( G \) with \( |X| \leq 2 \) (can = 0) and \( X \neq V(G) \). Then there is a vertex \( v \in V(G) - X \) such that \( \deg _{G} v \leq 2 \). \\
\textbf{Theorem 13.3.} If \( G \) is a loopless graph with no \( K_4 \) minor then \( \chi(G) \leq 3 \).
\section{Planar Graphs}
\textbf{Jordan Curve.} Any closed simple curve separates the plane into two regions. \\
\textbf{Lemma 14.1.} Let \( G \) be a graph drawn in the plane, \( e \in E(G) \). Then the regions on different sides of \( e \) are the same if and only if \( e \) is a cut-edge of \( G \). \\
\textbf{Euler's Formula.} Let \( G \) be a planar non-null graph. Then \[|V(G)| - |E(G)| + \operatorname{Reg}(G) = 1 + \operatorname{comp}(G) \] 
\textbf{Lemma 14.3.} Let \( G \) be a connected simple graph drawn in the plane with \( |E(G)| \geq 2 \). Then the length of every region of \( G \) is at least 3, and if it is 3 then the boundary is a cycle of length 3. \\
\textbf{Lemma 14.4.} If \( G \) is a simple planar graph, \( |E(G)| \geq 2 \), then \[|E(G)| \leq 3|V(G)| - 6\] and if \( G \) has no \( K_3 \) subgraph then \[|E(G)| \leq 2|V(G)| - 4.\]
\textbf{Fact:} \( K_5 \), \( K_{3,3}  \), and their subdivisions are the only minimal non-planar graphs. \\
\textbf{Corollary 14.5.} Let \( G \) be a simple planar graph, \( |E(G)| \geq 2 \), then \( \sum_{v \in V(G)}^{} (6 - \deg v)  \geq 12.\) \\
\textbf{Corollary 14.6.} If \( G \) is a simple non-null planar grapg then for some \( v \in V(G) \) \( \deg _{G} v \leq 5 \). Thus, planar graphs are \( 5 \)-degenerate so that \( \chi_{} (G) \leq 6 \).
\section{Kuratowski}
\textbf{Lemma 15.1.} Let \( G \) be a 2-connected loopless graph drawn in the plane. Then every region is bounded by a cycle. \\
\textbf{Lemma 15.2.} Let \( C \) be a cycle, \( X,Y \subseteq V(G) \), \( |V(C)| \geq 2 \), then one of the following holds:
\begin{enumerate}
	\item There exist \( z_1, z_2 \in V(C) \) distinct, two paths \( P,Q \) with ends \( z_1 \) and \( z_2 \) such that \( P \cup Q = C\), \( X \subseteq V(P) \) and \( Y \subseteq V(Q) \).
	\item There exist distinct \( x_1, x_2 \in X \) , \( y_1, y_2 \in Y \) such that \( x_1, y_1, x_2, y_2 \) appear on \( C \) in this order. 
	\item \( X = Y \) and \( |X|=|Y|=3 \).
	
	
	
\end{enumerate}
\textbf{Kuratowski-Wagner.} A graph \( G \) is non-planar if and only if either \( K_5 \) or \( K_{3,3}   \) is a minor of \( G \). \\
\textbf{EQUIVALENTLY:} A graph \( G \) is non-planar if and only if it contains a subdivision of \( K_5 \) or \( K_{3,3}  \) as a \textbf{subgraph}. \\
\textbf{Four color.} If \( G \) is planar and loopless then \( \chi_{} (G) \leq 4 \). \\
\textbf{Tait.} Let \( G \) be a planar triangulation and let \( G^{\ast}  \) be its dual. THen \( \chi_{} (G) \leq 4 \iff \chi_{} ' (G^{\ast} ) = 3 \). \\
\textbf{Kaufman.} For any pair of bracketings of the product \( u_1\times \cdots \times u_{m}  \) there is a choice of \( u_{n} \in \{ i,j,k \}  \) for every \( 1 \leq n \leq m \) such that the corresponding products are the same and non-zero.
\section{Definitions}
\begin{enumerate}
	\item \textbf{Matching.} A matching \( M \subseteq E(G) \) in a graph \( G \) is a collection of non-loop edges so that every vertex of \( G \) is incident to at most one edge of \( M \).
	\item \textbf{M-alternating path.} An \( M \)-alternating path \( P \) is a path whose edges alternate between belonging to \( M \) and belonging to \( E(G) \setminus M \). Equivalently, every internal vertex of the path is incident to one edge in \( M \cap E(P) \) and one edge of \( (E(G) \setminus M) \cap E(P). \)
	\item \textbf{M-augmenting path.} A path \( P \) is \( M \)-augmenting if \( |P| \geq 1 \), it is \( M \)-alternating, and the ends of \( P \) belong to no edge of \( M \) (whether in \( P \) or not).
	\item \textbf{Vertex cover.} A vertex cover is a subset \( X \subseteq V(G) \) if every edge of \( G \) has an end in \( X \).
	\item \( \nu (G) \) is the \textbf{matching number} of \( G \), i.e. the maximum size of a matching in \( G \). \( \tau (G) \) is the minimum size of a vertex cover in \( G \).
	\item \textbf{Cover.} A matching \( M \) covers \( Y \subseteq V(G) \) is every vertex in \( Y \) is incident to an edge of \( M \).
	\item \textbf{Perfect matching.} \( M \) is a perfect matching if it covers \( V(G) \).
	\item \textbf{Separation.} A pair \( (A,B) \), \( A,B \subseteq V(G) \) is called a separation if \( A \cup B = V(G) \) and no edge of \( G \) has one end in \( A \setminus B \) and another in \( B \setminus A \), i.e. any edge from \( A \) to \( B \) has an end in \( A \cap B \).
	\item \textbf{k-Connected.} A graph if \( k \)-connected if \( |V(G)| \geq k +1  \) and \( G \setminus X \) is connected for every \( X \subseteq V(G) \) with \( |X| \leq k+1. \)
	\item For \( X \subseteq V(G) \), let \( \delta (X) \) denote the set of all edges of \( G \) with one end in \( X \) and another in \( V(G) \setminus X \). Let \( N(X) \) the set of all neighbours of vertices in \( X \).
	\item \textbf{Line graph.} The line graph \( L(G) \) of \( G \) has \( V(L(G)) = E(G) \), and two vertices of \( L(G) \) are adjacent iff the edges they represent share an end in \( G \).
	\item \textbf{Directed graph.} A directed graph (digraph) is a graph where for every edge, one of its ends is chosen as a head and the other as a tail. An edge is said to be directed from its tail to its head.
	\item \textbf{Directed path.} A directed path from \( s \) to \( t \) in a digraph is a path in which every edge is traversed from its tail to its head as we follow the path from \( s \) to \( t \).
	\item For a set \( X \subseteq V(G) \), we let \( \delta ^{+} (X) \) denote the set of all edges with tail in \( X \) and head in \( V(G) \setminus X \); we let \( \delta ^{-} (X) = \delta ^{+} (V(G) \setminus X)  \).
	\item \textbf{(s,t)-Flow.} Let \( G \) be a digraph, \( s , t \in V(G) \). A function \( \varphi : E(G) \to \mathbb{R}^{+}  \) is an \( (s,t) \)-flow if for every \( v \in V(G) \setminus \{ s,t\}  \), \( \sum_{e \in \delta ^{+}(v) }^{}\varphi (e)= \sum_{e \in \delta ^{-} (v)}^{} \varphi (e) \). The \textbf{value} of \( \varphi  \) is \( \sum_{e \in \delta ^{+} (s)}^{}\varphi (e) - \sum_{e \in \delta ^{-} (s)}^{}\varphi (e). \)
\end{enumerate}
\section{Combinatorics}
\begin{enumerate}
	\item Max number of edges in simple graph on \( n \geq 1 \) vertices (i.e. \( |E(K_n)| \)) is \( \frac{n(n-1)}{2}  \).
	\item No formula for number of graphs up to isomorphism on \( n \) vertices; number of graphs up to isomorphism with \( n \) edges is \( 2^{\binom{n}{2} }  \).
	\item Number of spanning trees on \( n \) labelled vertices (i.e. in \( K_n \)) is \( n^{n-2}  \).
	\item Number of ways to label a path on \( n \) vertices: \( n! / 2 \) and a star on \( n \) vertices: \( n \).
	\item \( \nu(K_n) = \lfloor n / 2 \rfloor \), \( \tau (K_n) = n - 1 \), \( \nu (C_n) = \lfloor n / 2 \rfloor \), \( \tau (C_n) = \lceil n / 2 \rceil \).
	
	
	
	
	
\end{enumerate}
\section{Strategy}
\begin{enumerate}
	\item \textbf{Is \( G \) connected}: look at connected components and derive a contradiction.
	\item \textbf{Is \( G \) bipartite}: look at possible sizes of \( A,B \) in a bipartition, does this contradict the vertex degrees?
	\item \textbf{Is \( G \) 2-connected}:
	\item \textbf{Does \( G \) have an Euler tour}: yes if every vertex degree even (and must be connected)?
	\item \textbf{Does \( G \) have an Euler trail}: yes if \( G \) connected and has at most two vertices of odd degree. 
	\item \textbf{Does \( G \) have Hamiltonian cycle}:
		\begin{enumerate}
			\item No if there is non-null \( X \subseteq V(G) \) such that \( G \setminus X \) has more than \( |X| \) components;
			\item Yes if \( n \geq 3 \) vertices, and if for every pair of non-adjacent vertices \( \deg u + \deg v \geq |V(G)| = n \).
			\item Yes if \( |V(G)| \geq 3 \) and either (1) \( \deg v \geq n / 2 \) (2) \( |E(G)| \geq \binom{n}{2} - n + 3 \).
			
			
			
		\end{enumerate}
	\item \textbf{Does \( G \) have a perfect matching}:
		\begin{enumerate}
			\item Yes if bipartite and all vertices have degree \( d \geq 1 \).
			\item Yes iff \( \nu (G) = \frac{|V(G)|}{2}  \). If bipartite, this is equivalent to every vertex cover \( X \) is such that \( |X| \geq \frac{|V(G)|}{2}  \).
		\end{enumerate}
\end{enumerate}
\textbf{Useful inequalities}:
\begin{enumerate}
	\item If every vertex in \( G \) has degree at most \( d \), then \( d|X| \geq \sum_{v \in V(G)}^{}\deg v \geq |E(G)| \) 
	
\end{enumerate}
\end{multicols}
\end{document}

